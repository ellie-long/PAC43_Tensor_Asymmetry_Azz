\subsection{Interest from Theorists}

The measurement proposed has stirred interest in a number of theorists who either have provided or are currently working on calculations. Many of these are on-going and are expected to be completed in the coming year.

The light cone and virtual nucleon calculations of M.~Sargsian~\cite{misak-convo} and M.~Strikman~\cite{strikman-convo} are already available for $A_{zz}$ and are presented in this document. Calculations have been done with difference NN potentials and have found significant differences at large $x$.

Continuing his interest from DIS $b_1$ calculations~\cite{Cosyn:2014sqa}, W.~Cosyn has developed calculations of the quasi-elastic contribution to inclusive deuteron scattering, which will be the dominant contribution in the $x>1$ regime~\cite{cosyn-convo}. His calculations, which include final-state interactions, have been modified to include $A_{zz}$ and are presented in this proposal.

Although not completed at the time of submission, W. Van Orden has calculations in progress using different nucleon-nucleon potentials, as well as different prescriptions for handling the reactions mechanisms in the low $Q^2$ region for tensor polarization observables. In his words, ``Recent studies have shown that it is extremely challenging to disentangle reaction mechanisms from nucleon-nucleon potential effects using cross section information~\cite{Ford:2014yua}. This group is now in the process of extending their studies to vector and tensor asymmetries. Previous low $Q^2$ measurements seemed to indicate that the asymmetries are far less sensitive to reaction 
mechanisms than the cross sections~\cite{Passchier:2001uc}; so while the 
new calculations are not yet available, it is clear that the asymmetries will produce unique constraints 
on our understanding of the deuteron."~\cite{vanorden-convo}

G.~A.~Miller~\cite{miller-convo} has developed an interest in this measurement, and is working on calculations that involve 6-quark effects in the elastic region. In his own words, he states ``This proposal really challenges theorists to better understand the meaning of nuclear wave functions in a situation that demands a relativistic treatment. I plan on working to understand this reaction during the upcoming summer."

S.~Liuti also affirms the importance of understanding the structure of the deuteron in the kinematics presented in this proposal, stating ``This is an important measurement and should be calculated more thoroughly."~\cite{liuti-convo}


% Similar calculations have recently been done for the approved D($e,e'p)n$ at high $Q^2$, high $p_m$ experiment~\cite{Ford:2014yua}. 

In summary, we are encouraged that several theorists have been and continue to be engaged in serious efforts to calculate $A_{zz}$ in the $x>1$ region using a variety of models.