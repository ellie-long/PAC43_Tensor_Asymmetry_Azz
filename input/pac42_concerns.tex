\section{PAC42 Comments and Concerns}

In this section we summarize the comments and concerns that were raised by the PAC42 committees on letter of intent LOI12-14-002.

\subsection{Theory Advisory Committee}

\begin{quote}
``This Letter of Intent describes a measurement of the tensor-polarized asymmetry $A_{zz}$ in electron scattering on polarized deuterium in the quasi-elastic region, at values of $x = 0.8 - 1.75$ ($x$ is the equivalent Bjorken variable at the nucleon level) and $Q^2 = 1 − 2 \mathrm{~GeV} ^2$. The aim is to determine with this observable the $S/D$ wave ratio in the deuteron wave function at large relative momenta $k > 300$~MeV, which is important for understanding the $NN$ interaction at short distances and the properties of the dominant $pn$ short-range correlations in heavier nuclei. The same tensor-polarized asymmetry was/will be measured in elastic scattering (deuteron form factor) and deep-inelastic scattering (structure function $b_1$); the proposed measurement in quasi-elastic scattering would fill the gap and study this observable in the region where it is most directly related to the short-range $NN$ interaction. The tensor asymmetry at large recoil momenta also serves as a sensitive test of ``relativistic effects" in the treatment of deuteron structure, which are an important aspect of the overall theoretical framework and the object of ongoing studies. A unique feature of the measurement proposed here is that it selects small-size configurations in the deuteron both through the tensor asymmetry ($D$-state) and the choice of kinematics ($x > 1$), amplifying the overall effect. The use of $x > 1$ for selecting small-size $NN$ configurations has been demonstrated in previous studies of deep-inelastic structure. 

The measurement proposed here arises from a well-developed context, presents a clear objective, and enjoys strong theory support. It would further explore the nature of short-range $pn$ correlations in nuclei, the discovery of which has been one of the most important results of the JLab 6 GeV nuclear program. Development of a full proposal should be encouraged."
\end{quote}

\subsection{Technical Advisory Committee}

\begin{quote}
``This experiment utilizes the same apparatus and techniques as the conditionally approved b1 experiment C13-12-011. The comments in the TAC report for that experiment also apply to this experiment.

The requirement to understand and mitigate time-dependent systematic effects may be less as the asymmetry $A_{zz}$, at least for $x>1$, is expected to be larger than for b1. 
However, measuring with $\delta A_{zz} < 0.10$ still requires a systematic control of the raw asymmetry to better than 1\%. 
This is still challenging with a target polarization that is cycled on and off about once a day.
Furthermore, at $x>1$, short range structure enhances inclusive cross sections in nuclei relative to deuterium. This will reduce the dilution factor for $x>1$ measurements, reducing the raw asymmetries to levels where understanding and controlling systematic errors will still be important."
\end{quote}

\subsection{Program Advisory Committee}
\begin{quote}
``\textbf{Measurement and Feasibility:} Electron scattering off tensor-polarized deuterium would be measured in the quasi-elastic
region using the Hall C HMS and SHMS spectrometers. This proposal would use the same setup as the C1-approved
experiment E12-13-011, which is to measure the deuteron tensor structure function b1. The C1-approval is subject to
demonstration that 35\% tensor polarization is possible. The expected asymmetry is larger in the case of this measurement,
but we anticipate that a similar requirement would apply. It is anticipated that a full proposal would be for 39 days, which
would include 30 days for three different $Q^2$ values and 9.1 additional days of overhead.

\textbf{Issues:} A significant amount of beam time will be required for this measurement. A full proposal will need a detail
discussion of expected systematic and statistical errors similar to what is in the letter that carefully justifies the requested
time. The proposal should also demonstrate what sensitivity they will have to NN interaction models, such as the 6-quark
model, final state interaction models, and NN interaction models, mentioned in the proposal. It will also be important to
discuss how the results will distinguish between effects from the NN-interaction, the treatment of these interactions at
high virtuality, and the intrinsic deuteron wave function.

\textbf{Recommendation:} Proceed to proposal addressing the issues noted above."
\end{quote}

\subsection{Response to PAC42 Concerns}
The tensor polarization of 30\% used in the rates for this proposal is the same as condition on the E12-13-011 proposal, which was incorrectly mentioned as 35\%. To ensure that the target polarization is not significantly affected by the electron beam, we've reduced our estimated current from 90~nA in the LOI to \CURRENT~nA, which is the conservative standard for a DNP target. We have expanded upon our estimated statistical and systematic uncertainties in Section \ref{uncertainties} and in a recent technical note\need, where we stress that a measurement of $A_{zz}$ in the quasi-elastic and $x>1$ region is ideal for understanding time-dependent systematic effects without significantly affecting the measurement, as it is an order of magnitude less sensitive to drift effects than $b_1$. Furthermore, we will be measuring $T_{20}$ at low $Q^2$ where the observable is well understood both experimentally and theoretically, which can be used as a calibration to reduce our leading systematics from understanding the target polarization. There has also been a dedicated effort made in understanding the tensor-enhanced polarization state by the UVA group over the past few years. Through studying tensor polarization and NMR line-shape analysis through multiple cool-downs, the UVA group is confident that the uncertainty in polarization can be kept to $<6\%$~\cite{keller2, keller3}. However, even a very conservative estimate of $10\%$ would make for a compelling measurement.
%However, for the uncertainty calculations in this proposal we retain a conservative estimate of 12\%.


Since PAC42, we have engaged a number of theorists who have provided models not only between light cone and virtual nucleon models, but also using different $NN$ interaction potentials~\cite{Sargsian:2014fla} and final state interactions~\cite{cosyn-convo}. Deviations based on $NN$ potentials only become apparent at large $x>1.3$, so that the low $x<1.3$ region can be used to discriminate between light cone and virtual nucleon calculations. Furthermore, $A_{zz}$ calculations are currently being systematically studied at low $Q^2$ by W. Van Orden and are expected to be completed within a year~\cite{vanorden-convo}. Although not completed at the time of this proposal, G. Miller is still engaged in providing calculations of 6 quark effects in the elastic region~\cite{miller-convo}. Additionally, the proposed $A_{zz}$ measurements are also ideal for making simultaneous high-precision measurements of $T_{20}$ to test existing calculations up to large $Q^2$, as discussed in Section~\ref{t20_exp}, including in the region where Hall C and MIT-Bates data show a discrepancy, which requires only four more days of beam time than initially proposed in the LOI. Along with the ground-breaking measurements of $A_{zz}$ in the $x>1$ region, we will also be measuring $T_{20}$ in the largest $Q^2$ range ever taken in a single experiment.





