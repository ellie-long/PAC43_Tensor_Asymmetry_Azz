%For decades~\cite{PhysRev.81.165}, it has been known that the nucleon-nucleon potential has a short-range repulsive core, which is responsible for the stability of strongly interacting matter. However, a description of the repulsive core remains largely unconstrained and our understanding of QCD dynamics at short distances ($\leq 0.5\mathrm{~fm}$) largely incomplete~\cite{Sargsian:2014bwa}. 

The deuteron is the simplest composite nuclear system, and in many ways it is as important to understanding bound states in QCD as the hydrogen atom was to understanding bound systems in QED.  Our experimental and theoretical understanding of the deuteron remains unsatisfying, which we will address by measuring the tensor asymmetry $A_{zz}$ for the first time in a kinematic region where theoretical understanding of the deuteron is weak.

Due to their small size and simple structure, tensor polarized deuterons are ideal for studying nucleon-nucleon interactions. Tensor polarization enhances the D-state contribution, which compresses the deuteron~\cite{Forest:1996kp}, 
%in a toroid as shown in Fig.~\ref{fig:dpol-shape}, 
making the system more sensitive to short-range QCD effects. Understanding the nucleon-nucleon potential of the deuteron is essential for understanding short-range correlations as they are largely dependent on the tensor force~\cite{Arrington:2011xs}. We can resolve the short-range structure of nuclei on the level of nucleon and hadronic constituents by utilizing processes that transfer to the nucleon constituents both energy and momentum larger than the scale of the NN short-range correlations, particularly at $Q^2>1~(\mathrm{GeV}/c)^2$.


By taking a ratio of cross sections from electron scattering from tensor-polarized and unpolarized deuterons, 
\begin{equation}
A_{zz}=\frac{2}{fP_{zz}}\left(\frac{\sigma_p}{\sigma_u}-1\right),
\end{equation}
the S and D-wave states can be disentangled, leading to a fuller understanding of the repulsive nucleon core. A measurement of $A_{zz}$ is sensitive to the ratio $\frac{D^2-SD}{S^2+D^2}$ and it's evolution with increasing minimal momentum of the struck nucleon. Originally calculated by L.~Frankfurt and M.~Strikman~\cite{Frankfurt:1988nt}, this has recently been revisited by M.~Sargsian and M.~Strikman, who calculated $A_{zz}$ in this region using light cone and virtual nucleon approaches with multiple NN potentials~\cite{Sargsian:2014fla}. The calculations vary by up to a factor of 2, and can be experimentally determined at the $3-6\sigma$ level as discussed in this proposal. In this same region, effects from final state interactions have been calculated and are expected to have a significant effect at large $x$~\cite{cosyn-convo}.


For the lower $Q^2$ region, W. Van Orden has calculations in progress using different nucleon-nucleon potentials, as well as different prescriptions for handling the reaction mechanisms in tensor polarization observables in the low $Q^2$ region~\cite{vanorden-convo}. Although it is difficult to disentangle reaction mechanisms from NN potentials using cross section measurements, previous low $Q^2$ results have indicated that asymmetries are far less sensitive to the reaction mechanisms~\cite{Passchier:2001uc}. Similar calculations have recently been finalized for the D($e,e'p)n$ at high $Q^2$, high $p_m$ experiment~\cite{Ford:2014yua}. 

Additionally, measuring $A_{zz}$ in the quasi-elastic region will fill a gap in measurements performed on deuterium scattering. It is directly proportional to the elastic deuteron tensor analyzing powers by $A_{zz} = \sqrt{2} \left[ d_{20} T_{20} + d_{21} T_{21} + d_{22} T_{22}\right]$. Due to the large acceptance of the SHMS spectrometer, we will be taking data in the $x = 2$ elastic range as well, allowing us to measure $T_{20}$ at a large range in $Q^2$ (as contributions from $T_{21}$ and $T_{22}$ are small). In the deep inelastic region, $A_{zz}$ will soon be measured to extract the tensor structure function $b_1$ by the relation $A_{zz} \propto \frac{b_1}{F_1^D}$. Not only will measuring $A_{zz}$ in the quasi-elastic region provide information necessary for understanding the fundamental properties of the deuteron, but it will be the first experiment to bridge a gap in measurements of electron scattering from tensor-polarized deuterons. We emphasize that this measurement is pushing the limits of understanding the deuteron by going to kinematics where no no current measurements exist and where current theoretical understanding remains unsatisfying.



