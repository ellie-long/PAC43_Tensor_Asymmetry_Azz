
We request \production_days days of production beam time in order to  measure
the tensor asymmetry A$_{zz}$ and spin structure function $b_1$ using a 
longitudinally polarized deuteron 
%(\TARGET) 
target together with the Hall C HMS and SHMS spectrometers.
All existing theoretical predictions for $b_1$ in the region of interest predict small or vanishing
values for $b_1$  
%at intermediate values of $x$, 
in contrast to the apparent large negative result of the only existing measurement from HERMES. 
%Tensor structure measurements provide information not available from spin-1/2 targets. 

This experiment will provide access to the tensor quark polarization and allow a test of the
Close-Kumano sum rule, which vanishes in the absence of tensor polarization in the quark
sea.
Until now, tensor structure has been largely unexplored, so the study
of these quantities holds the potential of initiating a new field of spin physics at
Jefferson Lab.
%This measurement will help clarify the role quark orbital angular momentum plays in the nucleon spin, and open a new avenue of spin structure studies at Jefferson Lab.


