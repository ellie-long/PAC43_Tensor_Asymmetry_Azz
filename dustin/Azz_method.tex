%Format:latex
%\documentclass[fleqn,twoside]{article}
\documentclass[12pt]{article}
\setlength{\oddsidemargin}{0in}
\setlength{\evensidemargin}{0in}
\setlength{\textwidth}{6.0in}
\setlength{\topmargin}{0.0in}
\setlength{\textheight}{9.0in}

% if you want to include PostScript figures
\usepackage{graphicx}
% if you have landscape tables
%\usepackage[figuresright]{rotating}

% Personal packages
%\usepackage{dcolumn}
%\usepackage{epsfig}
%\usepackage{amsmath,amssymb,graphicx}
%\usepackage{rotate}

% declarations for front matter

\title{
Measurement of $A_{zz}$
}

\author{
dustin\\
}

\begin{document}
\maketitle

\begin{abstract}
A quick outline of how to measure $A_{zz}$.
\end{abstract}

\section{ Measurement of $A_{zz}$ }

The measured DIS double differential cross section for a spin-1 target characterized by a vector polarization $P_{z}$ and tensor polarization
$P_{zz}$ is expressed as,
\begin{equation}
\frac{d^2\sigma_p}{dxdQ^2}=\frac{d^2\sigma}{dxdQ^2}\left(1-P_zP_BA_1+\frac{1}{2}P_{zz}A_{zz}\right),
\label{eq:one}
\end{equation}
where, $\sigma_p$ ($\sigma$) is the polarized (unpolarized) cross section, $P_B$ is the incident electron beam polarization, and $A_1$ ($A_{zz}$) is the
vector (tensor) asymmetry of the virtual-photon deuteron cross section.  This allows us to write
the positive polarized tensor, $0<P_{zz}\leq 1$, asymmetry using unpolarized electron beam as,
\begin{equation}
\label{Azz}
A_{zz}=\frac{2}{P_{zz}}\left(\frac{\sigma^1-\sigma}{\sigma}\right),
\end{equation}
where $\sigma^1$ is the polarized cross section for
\begin{equation}
P_{zz}=\frac{n_+-2n_0+n_-}{n_++n_-+n_0},\mbox{ for~} n_++n_->2n_0.
\end{equation}
Here $n_m$ represents the portion of the ensemble in the $m$ state.

Using Eq. \ref{Azz} the asymmetry $A_{zz}$ compares two different cross sections measured under different 
polarization conditions of the target, positively tensor polarized and unpolarized.  
To obtain both relative cross section measurements in the same configuration the same target
cup and material will be used at alternating polarization states.  In addition the same exact field will be used to keep
acceptance consistent within the setability of the supper conducting magnet. 

\noindent [{\sl The NMR will be
used on both to probe polarization.  To move from polarized to unpolarized measurements the target
polarization will be annihilated using destructive NRM loop field changes and destructive DNP microwave pumping.
It is also possible to remove LHe in the nose of the target to remove the polarization by heating.
During unpolarized data taking the incident electron beam heating is enough to remove the thermal equilibrium polarization.

The NMR measurement will ensure zero polarization.  The target material will be
kept at $\sim$1 K for polarized and unpolarized data collection.  These
consistencies are used to minimize the systematic differences in the
polarized and unpolarized data collection.  To minimize systematic effect over 
time the polarization condition will be switched twice in a 24 hour period. 
This is expected to account for drift in integrated charge accumulation.

(I think we should move this discussion to another section dealing with target
physics and the overhead time accounting. Also,  I would favor dumping the LHe, and refilling the
nose.)}]

The expressions for the tensor asymmetry in Eq. \ref{Azz} needs to be modified to take into account the presence of
unpolarized nuclei in the deuterated ammonia ($^{14}$N$^2$H$_3$, ND$_3$ for 
short) target. Since many of the factors invoved in the cross sections cancel in
the ratio, the asymmetry can then be expressed in terms 
of the charge normalized, efficiency corrected numbers of polarized $N^1$ and 
unpolarized $N$ counts, 
\begin{equation} \label{3}
A_{zz}=\frac{2}{fP_{zz}}\left(\frac{N^1-N}{N}\right).
\end{equation}
Here $f$ is the dilution factor defined as,
\begin{equation}
f=\frac{N_D\sigma_D}{N_N\sigma_N+N_D\sigma_D+\Sigma N_A\sigma_A},
\end{equation}
where $N_D$ is the number of deuterium nuclei in the target and and $\sigma_D$ 
is the corresponding inclusive double differential scattering cross 
section,
$N_N$ is the nitrogen number of scattered nuclei with cross section $\sigma_N$, and
$N_A$ is the numbers of other scattering nuclei of mass number $A$ with cross section $\sigma_A$.
The denominator of the dilution factor can be written in terms of the relative volume ratio of
$ND_3$ to $LHe$ in the target cell, or the packing fraction $p_f$.  In our case of cylindrical geometry
the packing fraction is equivalent to the percent of the cell length filled with $ND_3$.  For the
full development of the dilution factor see Appendix \ref{dil}.

The measurement of the tensor asymmetry allows for a calculation of tensor structure function $b_1$
using the world data on the leading-twist structure function $F_1^d$,
\begin{equation}
b_1=-\frac{3}{2}A_{zz}F_1^d.
\end{equation} 

In addition $b_1$ can be calculated directly using the difference of the two 
measured cross sections, however the uncertainties will be larger 
than for $A_{zz}$.

The time necessary to achieve the desired precision $\delta A$ is:
\begin{equation}
T=\frac{N_T}{R_D}=\frac{16}{P_{zz}^2f^2\delta A_{zz}^2R_D}.
\end{equation} 
where $R_D$ is the deuteron rate and $N_T = N^1 + N$ is the total estimated 
number of counts to achieve the uncertainty $\delta A_{zz}$.  See Appendix
\ref{stat} for full details of the statistical uncertainty.

\section{ Systematic Uncertainty in $A_{zz}$ }
The systematic uncertainty of the asymmetry $A_{zz}$ can be estimated based on know relative uncertainties and 
the systematic effects seen in past experiments.  

\subsection{ Target Polarization }
The target positive tensor polarization $P_{zz}$  is calculated using the vector polarization $P_{z}$
using Boltzmann statistics for spin temperature equilibrium,
\begin{equation}
P_{zz}=2-\sqrt{4-3P_z^2}.
\end{equation}
The uncertainty in $P_{zz}$ depends only on the uncertainty in the NMR measurement of $P_z$.
This leads to the expression,
\begin{equation}
\delta P_{zz}=\frac{3P_z}{\sqrt{4-3P_z^2}}\delta P_z.
\end{equation}
Polarization uncertainty for $ND_4$ have historically been no smaller than 5\%.  However with
new techniques in polarization uncertainty minimization we anticipate to be able to achieve
considerable reduction.  Here we use the estimate of 4\% relative uncertainty in $P_z$ for and
average vector polarization of 45\% leading to a relative uncertainty in $P_{zz}$ of 7.7\%.

\subsection{ Time dependent factors }

Systematic variation in time due to detector drift was
studied for transversity JLab experiment E06-010. For 3 months running, all detectors in
HRS were stable to about a 1\% level.  The scintillators, drift chambers, and 
lead-glass shower detector are stable to $\sim$2\% in 3 months, assuming
no significant radiation damage or detector gas loss.
For the measurement of $A_{zz}$ we expect no issue with radiation damage
being the beam current is comparatively low and in the spectrometer.

\subsection{ Radiative Corrections }
The systematic effect on $A_{zz}$ due to the QED radiative corrections will be quite small. 
Based on previous data for unpolarized radiative corrections we use a 1.5\% uncertainty.
The polarized contribution is considered to be negligible for the range in $x$ that we
are measuring.

\subsection{ Charge Determination }
The Beam Charge Monitor at low current are estimated to have an uncertainty 
lower than 5\%.  Integrating over a reasonable time the charge can be measured to 
approximately 1\%.  The Hall A tungsten calorimeter can be used to further reduce this uncertainty.

\subsection{ Total Systematic Uncertainty }
Table \ref{error1} shows a list of the leading uncertainties contributing to the systematic error in $A_{zz}$.
The resulting estimate in the relative uncertainty of $A_{zz}$ is 9.2\%.
\begin{table}
\begin{center}
\begin{tabular}{lccc}
(\#)&source & error (\%)\\ \hline  
(1)&Target Polarization & 8\% \\
(2)&Dilution/Packing fraction & 4\% \\
(3)&Detector Drift & 1\% \\
(4)&Radiative Corrections & 1.5\% \\
(5)&Charge Determination & 1\% \\
(6)&Detector resolution and efficiency & 1\% \\\hline
   &Total & 9.2\% \\\hline
\end{tabular}
\end{center}
\caption{The systematic error estimates of the $A_{zz}$ asymmetry measurement.}
\label{error1}
\end{table}

\appendix
\section{Rendering Dilution Factor}
\label{dil}
To derive the dilution factor we first start with the ratio of 
polarized to unpolarized counts.
%equation used to obtain the observable in terms of each measured cross section.
%\begin{equation}
%\frac{A_{zz}P_{zz}}{2}=\left(\frac{\sigma^1-\sigma}{\sigma}\right).
%\end{equation}
In each case, the number of counts that are actually measured, and neglecting 
the small contributions of the thin aluminium cup window materials, NMR coils, etc.,
are
\begin{equation}
N_1=Q_1\varepsilon_1 {\cal A}_1 l_1[(\sigma_N+3\sigma_1)p_f+\sigma_{He}(1-p_f)],
\end{equation}
and
\begin{equation}
N=Q\varepsilon {\cal A}l[(\sigma_N+3\sigma)p_f+\sigma_{He}(1-p_f)].
\end{equation}
where $Q$ represents accumulated charge, $\varepsilon$ is the dectector 
efficiency, ${\cal A}$ the cup acceptance, and $l$ the cup length.  

For
this calculation we assume similar charge accumulation such that $Q\simeq Q_1$, 
and that the efficiencies stay constant, in which case all factors drop out of 
the ratio leading to
\begin{eqnarray}
\nonumber \frac{N_1}{N}& = &\frac{{(\sigma_N+3\sigma_1)p_f+\sigma_{He}(1-p_f)}
}{(\sigma_N+3\sigma)p_f+\sigma_{He}(1-p_f)}\\
\nonumber & = & \frac{{(\sigma_N+3\sigma(1+2A_{zz}P_{zz}/2))p_f+\sigma_{He}(1-p_
f)}}{(\sigma_N+3\sigma)p_f+\sigma_{He}(1-p_f)}\\
\nonumber & = & \frac{{[(\sigma_N+3\sigma)p_f+\sigma_{He}(1-p_
f)]+3\sigma A_{zz}P_{zz}/2}}{(\sigma_N+3\sigma)p_f+\sigma_{He}(1-p_f)}\\
\nonumber & = & 1 + \frac{3\sigma 
A_{zz}P_{zz}/2}{(\sigma_N+3\sigma)p_f+\sigma_{He}(1-p_f)}\\
& = & 1 + \frac{1}{2} f A_{zz}P_{zz}, 
\end{eqnarray}
where $\sigma_1 = \sigma(1+2A_{zz}P_{zz}/2)$ has ben substituted, per 
eq.~(\ref{eq:one}), with $P_B =0$. It can be seen that the above result 
corresponds to eq.~(\ref{3}) in the main text.

\section{Statistical Uncertainty Calculation}
\label{stat}
To investigate the statistical uncertainty we start with the equation for $A_{zz}$ using
measured counts for polarized data $N_1$ and unpolarized data $N$, 
\begin{equation}
A_{zz}=\frac{2}{fP_{zz}}\left(\frac{N_1-N}{N}\right).
\end{equation}
The absolute error with respect to counts in then,
\begin{equation}
\delta A_{zz}=\frac{2}{fP_{zz}}\sqrt{\left(\frac{\delta N_1}{N}\right)^2+\left(\frac{N_1\delta N}{N^2}\right)^2}.
\end{equation}
To approximate, assume $N_1\simeq N$, so that twice $N$ is required to obtain the total number of count
$N_T$ for the experiment leading to,
\begin{equation}
\delta A_{zz}=\frac{4}{fP_{zz}}\frac{1}{\sqrt{N_T}}.
\end{equation}


\clearpage



\end{document}

\appendix
\section{Rendering Dilution Factor}
\label{dil}
To acquire the dilution factor we first start with the equation used
to obtain the observable in terms of each measured cross section.
\begin{equation}
\frac{A_{zz}P_{zz}}{2}=\left(\frac{\sigma^1-\sigma}{\sigma}\right).
\end{equation}
In each case the number of counts that are actually measured is,
\begin{equation}
N_1=Q_1A_1l_1[(\sigma_N+3\sigma_1)p_f+\sigma_{He}(1-p_f)],
\end{equation}
and
\begin{equation}
N=QAl[(\sigma_N+3\sigma)p_f+\sigma_{He}(1-p_f)].
\end{equation}
Here $Q$ represents accumulated charge, $A$ the cup acceptance, and $l$ the cup length.  For
this calculation we assume similar charge accumulation such that $Q\simeq Q_1$ in which case all
factors drop out of the ratio leading to,
\begin{equation}
\frac{N_1-N}{N}=\frac{{(\sigma_N+3\sigma_1)p_f+\sigma_{He}(1-p_f)}-[(\sigma_N+3\sigma)p_f+\sigma_{He}(1-p_f)]}{(\sigma_N+3\sigma)p_f+\sigma_{He}(1-p_f)}.
\end{equation}
\begin{equation}
\frac{N_1-N}{N}=\frac{{(\sigma_N+3\sigma(1+1/2A_{zz}P_{zz}))p_f+\sigma_{He}(1-p_f)}-[(\sigma_N+3\sigma)p_f+\sigma_{He}(1-p_f)]}{(\sigma_N+3\sigma)p_f+\sigma_{He}(1-p_f)}.
\end{equation}
\begin{equation}
\frac{N_1-N}{N}=\frac{1/2A_{zz}P_{zz}p_f3\sigma}{(\sigma_N+3\sigma)p_f+\sigma_{He}(1-p_f)}=\frac{A_{zz}P_{zz}f}{2}.
\end{equation}

