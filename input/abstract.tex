In this update to LOI12-14-002, we propose the first measurement of the tensor asymmetry $A_{zz}$ in the quasi-elastic region through the tensor polarized D($e,e'$)X channel; an asymmetry that is sensitive to the nucleon-nucleon potential.  Previous measurements of $A_{zz}$ have been used to extract $b_1$ in the DIS region and $T_{20}$ in the elastic region. In the quasi-elastic region, $A_{zz}$ data will be used to compare light cone calculations with variation nucleon-nucleon calculations, and is an important quantity to determine for understanding tensor effects, such as the dominance of $pn$ correlations in nuclei.

$A_{zz}$ was first calculated in 1988 by Frankfurt and Strikman, using the Hamada-Johnstone and Reid soft-core wave functions~\cite{Frankfurt:1988nt}. Recent calculations by
M. Sargsian and M. Strikman revisit $A_{zz}$ in the $x > 1$ range using virtual-nucleon and light-cone methods, which differ by up to a factor of two~\cite{MISAK} and can be discriminated experimentally at the $3-6\sigma$ level. Additionally, the calculations were done using multiple deuteron wavefunctions which show a large discrepancy at high $x$.

Additionally, our measurement of $A_{zz}$ allows for a simultaneous measurement of the tensor analyzing power $T_{20}$ without any further beam time or equipment by making a kinematic cut on the elastic peak. The lowest $Q^2$ measurement will fall on the most experimentally probed and theoretically understood region, making it ideal for ensuring that the tensor polarized target is operating correctly and to help reduce target systematic uncertainty, the leading systematic in this experiment. At medium $Q^2$, our measurement will fall in the same region where there is currently a discrepancy between Hall A and Hall C results. Our final point will lie at the highest $Q^2$ value ever measured for $T_{20}$, and will test whether the observable continues on a plateau. These measurements of $T_{20}$ will also cover the largest range in $Q^2$ measured by a single experiment.

We propose an experimental determination of $A_{zz}$ and $T_{20}$ utilizing the same equipment as the E13-12-011 $b_1$ experiment.  Three different $Q^2$ values will be measured over the course of \productiondays days, with \overheaddays additional days of overhead. The measurements are less sensitive to systematic uncertainties than E13-12-011, so this experiment would be utilized in parallel to better understand the in-beam conditions and time-dependent systematic effects of a tensor polarized target for the $b_1$ experiment.

Additionally, the kinematic requirements of $A_{zz}$ also allow for the simultaneous measurement of $T_{20}$ at multiple $Q^2$ points ranging from $0.13<Q^2<1.8$~GeV$^2$, the largest range of a single experiment. At low $Q^2\sim0.15$~GeV$^2$, $t_{20}$ is well known experimentally and theoretically making it an ideal calibration point to reduce systematics from target polarization. At mid $Q^2\sim0.7$~GeV$^2$, we will take a measurement in the region where recent JLab Hall C data disagrees with data from Hall A. Finally, at high $Q^2\sim1.8$~GeV$^2$ we will measure $T_{20}$ at the largest momentum transfer yet, further constraining calculations at high $Q^2$.






%We propose the first measurement of the tensor asymmetry $A_{zz}$ in the quasi-elastic region through the $\stackrel{\leftrightarrow}{\mathrm{D}}$($e,e'$)X channel to determine information on the tensor portion of the deuteron wavefunction. Previous measurements of $A_{zz}$ have been used to extract $b_1$ in the DIS region and $T_{20}$ in the elastic region. In the quasi-elastic region,  $A_{zz}$ can be used to extract the ratio of the S and D-states in the deuteron wave function. This ratio is currently not well constrained experimentally and is an important quantity to determine for understanding tensor effects, such as NN short range correlations, and is most clearly manifested in the scattering off the polarized deuteron due to a strong dependence of the S/D ratio on the nucleon momentum.

%In the quasi-elastic region, $A_{zz}$ was first calculated in 1988 by Frankfurt and Strikman, using the Hamada-Johnstone and Reid soft-core wave functions \cite{Frankfurt:1988nt}. Recent calculations by {M.~Sargsian} revisit $A_{zz}$ in the $x>1$ range using virtual-nucleon and light-cone methods, which differ by up to a factor of two \cite{MISAK} and can be discriminated experimentally at the $3-6 \sigma$ level.

%An experimental determination of $A_{zz}$ could be performed utilizing identical equipment identical as the E13-12-011 $b_1$ experiment at three different $Q^2$ values over the course of \productiondays days, with \overheaddays additional days of overhead. The measurements are less sensitive to systematic uncertainties than E13-12-011, such that this experiment could additionally be utilized to understand the in-beam conditions and time-dependent systematic effects of a tensor polarized target.
