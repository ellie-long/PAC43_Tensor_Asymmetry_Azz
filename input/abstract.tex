In this update to LOI12-14-002, we propose the first measurement of the tensor asymmetry $A_{zz}$ in the quasi-elastic region through the tensor polarized D($e,e'$)X channel; an asymmetry that is sensitive to the nucleon-nucleon potential and the deuteron wave function at short distances.  Previous measurements of $A_{zz}$ have been used to extract $b_1$ in the DIS region and $T_{20}$ in the elastic region. In the quasi-elastic region, $A_{zz}$ will be used to advance our understanding of the relativistic dynamics of the bound $pn$ system, as well as the nature of $pn$ interactions at short distances. A unique feature of the measurements proposed here is that it selects small-size configurations both through the tensor asymmetry, which enhances the D-state, and the choice of $x>1$ kinematics. Taken together, these features amplify short-range effects. The possibility to measure tensor observables with high precision at Jefferson Lab has attracted the attention of a number of theorists, which have provided a range of predictions that require a measurement of $A_{zz}$ to constrain. 
In the words the PAC42 theory review, ``The measurement proposed here arises from a well-developed context, presents a clear objective, and enjoys strong theory support. It would further explore the nature of short-range $pn$ correlations in nuclei, the discovery of which has been one of the msot important results of the JLab 6~GeV nuclear program."
%The measurements described in this proposal push the limits of understanding the deuteron by entering kinematics that haven't been measured and where current theoretical understanding is weak.

%data will be used to compare light cone calculations with virtual nucleon-nucleon calculations, and is an important quantity to determine for understanding tensor effects, such as the dominance of $pn$ correlations in nuclei.

Within the past few years, there has been a strong theoretical effort to better understand the deuteron through the quasi-elastic $A_{zz}$ observable. First calculated in 1970's to demonstrate its usefulness in probing the short-range nuclear core in the region where the D-state dominates, it has recently been revisited by
M.~Sargsian and M.~Strikman who have calculated $A_{zz}$ in the $x > 1$ range using light-cone and virtual-nucleon methods. Probing these relativistic models is crucial for advancing our understanding of short range correlations. For quasi-elastic $A_{zz}$, these methods differ by up to a factor of two~\cite{MISAK} and can be discriminated experimentally at the $3-6\sigma$ level. Additionally, the calculations are sensitive to the S/D ratio of the input deuteron wave functions at large relative momenta $k>300$~MeV, which show a large discrepancy at high $x$ and can be used to further the original intent of $A_{zz}$ to discriminate between soft and hard $pn$ potentials. 

A number of other theoretical efforts have begun to further use $A_{zz}$ to understand the simplest composite nuclear system. Effects from final state interactions have been calculated using a virtual nucleon model with various NN potentials by W.~Cosyn, which are expected to have a significant effect on $A_{zz}$ at large $x$ and must be understood to probe nucleonic effects. Calculations on 6-quark effects in the elastic region will soon be calculated by G.A. Miller. Low $Q^2$ calculations are currently in progress by W. Van Orden, who has shown that asymmetry measurements are less sensitive to reaction mechanisms than cross sections, allowing them to be disentangled from NN potential effects. 

%Additionally, our measurement of $A_{zz}$ allows for a simultaneous measurement of the tensor analyzing power $T_{20}$ without any further beam time or equipment by making a kinematic cut on the elastic peak. The lowest $Q^2$ measurement will fall on the most experimentally probed and theoretically understood region, making it ideal for ensuring that the tensor polarized target is operating correctly and to help reduce target systematic uncertainty, the leading systematic in this experiment. At medium $Q^2$, our measurement will fall in the same region where there is currently a discrepancy between Hall C and MIT-Bates results. Our final point will lie at the highest $Q^2$ value ever measured for $T_{20}$, and will test whether the observable continues on a plateau. These measurements of $T_{20}$ will also cover the largest range in $Q^2$ measured by a single experiment.

The kinematic requirements of $A_{zz}$ allow for the simultaneous measurement of elastic $T_{20}$ at multiple $Q^2$ points ranging from $0.2<Q^2<1.8$~GeV$^2$, the largest range ever for a single $T_{20}$ experiment. At low $Q^2\sim0.2$~GeV$^2$, $T_{20}$ is well known experimentally and theoretically making it an ideal calibration point to reduce uncertainty from target polarization. At mid $Q^2\sim0.7$~GeV$^2$, we will take a high-precision measurement in the region where previous JLab Hall C results systematically disagree with results from MIT-Bates. Finally, at high $Q^2\sim1.5$ and $1.8$~GeV$^2$ we will measure $T_{20}$ up to the largest momentum transfer yet while providing a crucial check of the only existing data at large $Q^2>1$, which systematically differs from other results at lower $Q^2$.

One of the leading uncertainties that will effect all tensor polarized deuterium experiments is the absolute knowledge of the tensor polarization.   We have found that by using the elastic reaction at low $Q^2$, we can normalize our target's degree of polarization to the high precision and low $Q^2$ NIKHEF measurement.  The NIKHEF tensor polarization was created with an atomic beam source and they were able to measure the polarization of the gas with both a Briet-Rabi polarimeter as well as an ion-extraction polarimeter. 

We propose an experimental determination of $A_{zz}$ and $T_{20}$ utilizing the same equipment as the E13-12-011 $b_1$ experiment.  Six different $Q^2$ values of $A_{zz}$ and four of $T_{20}$ will be measured over the course of \productiondays days, with \overheaddays additional days of overhead. The proposed $A_{zz}$ measurements are more than an order of magnitude less sensitive to systematic uncertainties than E13-12-011, so this experiment could also be utilized to better understand the in-beam conditions and time-dependent systematic effects of a tensor polarized target for the $b_1$ experiment. This experiment will play a crucial role in the larger tensor program at Jefferson Lab, which continues to generate interest from experimental, polarized target, and theoretical spin communities, by providing the first experimental data in an region where there remains a gap in our understanding of the simplest composite nuclear system.








%We propose the first measurement of the tensor asymmetry $A_{zz}$ in the quasi-elastic region through the $\stackrel{\leftrightarrow}{\mathrm{D}}$($e,e'$)X channel to determine information on the tensor portion of the deuteron wavefunction. Previous measurements of $A_{zz}$ have been used to extract $b_1$ in the DIS region and $T_{20}$ in the elastic region. In the quasi-elastic region,  $A_{zz}$ can be used to extract the ratio of the S and D-states in the deuteron wave function. This ratio is currently not well constrained experimentally and is an important quantity to determine for understanding tensor effects, such as NN short range correlations, and is most clearly manifested in the scattering off the polarized deuteron due to a strong dependence of the S/D ratio on the nucleon momentum.

%In the quasi-elastic region, $A_{zz}$ was first calculated in 1988 by Frankfurt and Strikman, using the Hamada-Johnstone and Reid soft-core wave functions \cite{Frankfurt:1988nt}. Recent calculations by {M.~Sargsian} revisit $A_{zz}$ in the $x>1$ range using virtual-nucleon and light-cone methods, which differ by up to a factor of two \cite{MISAK} and can be discriminated experimentally at the $3-6 \sigma$ level.

%An experimental determination of $A_{zz}$ could be performed utilizing identical equipment identical as the E13-12-011 $b_1$ experiment at three different $Q^2$ values over the course of \productiondays days, with \overheaddays additional days of overhead. The measurements are less sensitive to systematic uncertainties than E13-12-011, such that this experiment could additionally be utilized to understand the in-beam conditions and time-dependent systematic effects of a tensor polarized target.
