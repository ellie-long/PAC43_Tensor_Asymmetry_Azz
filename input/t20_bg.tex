
The cross section of unpolarized elastic electron-deuteron scattering (assuming one-photon-exchange) is determined by
\begin{equation}
\frac{d\sigma}{d\Omega} = \left. \frac{d\sigma}{d\Omega}\right\vert_{NS}\left[A(Q^2)  + B(Q^2)\tan^2\frac{\theta}{2}  \right],
\end{equation}
where $A$ and $B$ are related to the charge ($G_C$), magnetic ($G_M$), and quadrupole ($G_Q$) form factors by
\begin{equation}
A=G_C^2(Q^2)+\frac{8}{9} \eta ^2 G_Q^2 + \frac{2}{3} \eta G_M^2(Q^2),
\end{equation}
\begin{equation}
B=\frac{4}{3}\eta (1+\eta )G_M^2(Q^2).
\end{equation}
In order to separate out all three form factors, a tensor polarized measurement is also needed. Although a number of tensor analyzing powers are available, $T_{20}$ has proven to be the most informative and has been studied more indepth than the others. This analyzing power is defined by
\begin{equation}
T_{20} = -\frac{\frac{8}{9}\eta^2 G^2_Q + \frac{8}{3} \eta G_C G_q + \frac{2}{3} \eta G_M^2\left[\frac{1}{2} + (1 + \eta) \tan^2(\theta / 2) \right]}{\sqrt{2} [A + B\tan^2(\theta / 2)]},
\end{equation}
where $\eta=Q^2/4M^2$, and can be measured by knowing either the initial or final polarization state.


%For the following discussion we use the nomenclature described by the Madison Convention~\cite{MadisonConv}.

