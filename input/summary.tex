\section{Summary}


We have investigated the possibility of making high precision measurements of the quasi-elastic tensor asymmetry $A_{zz}$.  By covering the kinematic range from the QE peak ($x\approx 1$) up to elastic scattering ($x=2$), we expect that this data will provide valuable new insights about the high momentum components of the deuteron wavefunction. We have been actively working with several theorists who have provided state-of-the-art calculations of light cone, virtual nucleon, and final state interactions. Additional calculations are being performed that include six-quark models, and low $Q^2$ sensitivity to NN potentials.  It is important to note that this is the same kinematic region that has been shown to be correlated with the EMC effect via the $x>1$ A/D ($e,e'$) results. 

Additionally, our measurement of $A_{zz}$ allows for a simultaneous measurement of the tensor analyzing power $T_{20}$ without any further beam time or equipment by making a kinematic cut on the elastic peak. The lowest $Q^2$ measurement will fall on the most experimentally probed and theoretically understood region, making it ideal for ensuring that the tensor polarized target is operating correctly and to help reduce target systematic uncertainty, the leading systematic in this experiment. At medium $Q^2$, our measurement will fall in the same region where there is currently a discrepancy between Hall C and Bates results. Our final point will lie at the highest $Q^2$ value ever measured for $T_{20}$, and will provide a crucial test of ensuring our understanding of $T_{20}$. These measurements of $T_{20}$ will also cover the largest range in $Q^2$ measured by a single experiment.


We have found that with \productiondays days of beam and an additional \overheaddays days of overhead, $A_{zz}$ can be measured with high precision at $Q^2=0.2$, 0.3, 0.7, 1.5, 1.8 and $2.9~(\mathrm{GeV}/c)^2$ and $T_{20}$ at $Q^2=0.2$, 0.3, 0.7, 1.5, and $1.8~(\mathrm{GeV}/c)^2$ in Hall C using identical equipment as the upcoming $b_1$ measurement while being orders of magnitude less sensitive to systematic uncertainties. In addition, this data will fill a gap in measurements of $A_{zz}$ between the $T_{20}\propto A_{zz}$ elastic measurements and the $b_1\propto \frac{A_{zz}}{F_1^d}$ deep-inelastic measurements. 

% -----------------------------------------------------

