\subsection{$T_{20}$ Experimental Method} %Measurement of $A_{zz}$ }
\label{t20_exp}

A measurement of $T_{20}$ will be extracted from $A_{zz}$ on the elastic peak for each $Q^2$ mentioned in Section~\ref{kinematics}. We will follow a similar method to B. Boden \emph{et al}~\cite{Boden:1990una}, which also measured $T_{20}$ using a tensor polarized target. Our methods differ only in that we will use the high resolution of the HMS and SHMS to determine the elastic peak through kinematic cuts, where Boden utilized a second spectrometer for that purpose. 

Boden describes the relationship between the measured ratio of unpolarized and tensor-polarized yields and $T_{20}$ in the following manner. They start with the experimental ratio $R_{exp}$ between the tensor polarized and unpolarized states,

\begin{equation}
R_{exp}=\frac{d\sigma_p}{d\sigma_u}=1+P_{zz} \left( \frac{d\sigma_T-d\sigma_u}{d\sigma_u} \right) = 1+P_{zz} \left( R_T-1 \right) 
\end{equation}

\begin{equation}
R_T-1 = \frac{1}{P_{zz}} \left( \frac{d\sigma_p}{d\sigma_u} -1 \right).
\end{equation}

This matches closely to our definition of $A_{zz}$,

\begin{equation}
A_{zz}=\frac{2}{fP_{zz}}\left( \frac{d\sigma_p}{d\sigma_u}-1\right), \end{equation}

where the dilution factor would be $f=1$ in the Boden extraction, since they are tagging deuterons and have negligible contamination from other sources.

Thus, $R_T$ and $A_{zz}$ are related by

\begin{equation}
R_T-1=\frac{1}{2}A_{zz}.
\end{equation}

Utilizing forward electron scattering to reduce magnetic contributions to a negligible amount, $(R_T-1)$ is related to $T_{20}$ by

\begin{equation}
T_{20}=-\sqrt{8}(R_T-1),
\end{equation}

which we can use to relate $T_{20}$ to an elastic measurement of $A_{zz}$ by

\begin{equation}
T_{20}=-\sqrt{2}A_{zz}.
\end{equation}


